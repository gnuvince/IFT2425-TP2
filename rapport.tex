\documentclass[10pt]{article}

\usepackage[utf8]{inputenc}
\usepackage[french]{babel}
\usepackage{amsmath}
\usepackage{amsfonts}
\usepackage{amssymb}
\usepackage{graphicx}
\usepackage{parskip}


\begin{document}

\title{IFT2425 - TP2 - Rapport}
\date{Mars 2011}
\author{Vincent Foley-Bourgon (FOLV08078309) \and
  Eric Thivierge (THIE09016601)}

\maketitle

\section{Problème et solution}

Nous faisons la résolution d'un système d'équations linéaires en
utilisant deux méthodes itératives, Jacobi et Gauss-Seidel.  On
s'intéresse à leur vitesse de convergence ainsi qu'à l'exactitude du
résultat qu'elles offrent.

\section{Réponses aux questions}

\subsection{Question 1}

Voici la matrice $A$ et le vecteur $\vec{b}$ étant donné les
paramètres suivants:

\begin{itemize}
\item $N = 5$: dimensions de la matrice
\item $M = 0.3$: valeur à mettre dans la matrice
\end{itemize}

\[
A_{N \times N} = \begin{bmatrix}
  1.00 & 0.30 & 0.09 & 0.00 & 0.00 \\
  0.30 & 1.00 & 0.30 & 0.09 & 0.00 \\
  0.09 & 0.30 & 1.00 & 0.30 & 0.09 \\
  0.00 & 0.09 & 0.30 & 1.00 & 0.30 \\
  0.00 & 0.00 & 0.09 & 0.30 & 1.00
\end{bmatrix}
\vec{b} = \begin{bmatrix}
  1.39 \\
  1.69 \\
  1.78 \\
  1.69 \\
  1.39
\end{bmatrix}
\]

\subsection{Question 2}

La matrice $A$ est diagonalement dominante, car:

\[
|a_{ii}| > \left|\sum_{j=1, j \ne i}^5a_{ij}\right| \quad\quad i=1\ldots 5
\]

La fonction \emph{DiagonallyDominant} du programme C permet de
vérifier qu'une matrice carrée est diagonalement dominante.

\subsection{Question 3}

Comme la matrice $A$ est diagonalement dominante, on peut résoudre ce
système d'équations à l'aide des méthodes de Jacobi et de
Gauss-Seidel.  Ces deux méthodes sont respectivement implantées dans
les fonctions C $SolveJacobi$ et $SolveGaussSeidel$; elles prennent
trois paramètres formels, la matrice $A$, le vecteur $\vec{b}$ et un
vecteur $\vec{x}$ dans lequel sera écrite la solution.  Les deux
fonctions retournent le nombre d'itérations requis pour résoudre le
système.

En effectuant la résolution avec un paramètre $\epsilon = 10^{-10}$,
on trouve avec les deux méthodes la solution suivante:

\[
\vec{x} = \begin{bmatrix}
  1 \\ 1 \\ 1 \\ 1 \\ 1
\end{bmatrix}
\]


\subsection{Question 4}

Le programme a résolu le système en 39 itérations avec la méthode de
Jacobi et en 12 itérations avec la méthode de Gauss-Seidel.  Ceci
s'explique par le fait que la méthode de Gauss-Seidel utilise toujours
les approximations les plus exactes connu à un moment donné pour
effectuer ses calculs, alors que la méthode de Jacobi ignore les
valeurs plus exactes jusqu'à la prochaine itération. De ce fait, la
méthode de Gauss-Seidel converge plus rapidement.

Sur de grands systèmes parallèles, la méthode de Jacobi pourrait être
avantageuse, car elle est plus facilement parallélisable (moins de
conflits de course), mais pour une résolution sur de petits systèmes,
comme par exemple des téléphones cellulaires, il est préférable de
choisir la méthode de Gauss-Seidel.

\subsection{Question 5}

Dans le cas où on fixe le paramètre $M=0.5$, on obtient les résultats
suivants:

\[
\vec{x}_{\text{Gauss-Seidel}} = \begin{bmatrix}
  1 \\ 1 \\ 1 \\ 1 \\ 1
\end{bmatrix} \tag{Itérations: 20}
\]

\[
\vec{x}_{\text{Jacobi}} = \begin{bmatrix}
  65874949127363408333634736203520540672 \\
  97324716552520863844231982403670769664 \\
  111950605506629327206985432673032339456 \\
  97324716552520863844231982403670769664 \\
  65874954197965809246552342190333362176
\end{bmatrix} \tag{Itérations: 578}
\]

La méthode de Gauss-Seidel donne encore la bonne solution, mais on
voit qu'il y a eu une explosion d'erreur numérique du côté de Jacobi;
elle a donné une solution avec un facteur d'erreur astronomique et a
pris près de 30 fois plus d'itérations que Gauss-Seidel pour arriver à
ce mauvais résultat.

Définitivement, pour résoudre ce système, on voudrait utiliser la
méthode de Gauss-Seidel.

\end{document}
